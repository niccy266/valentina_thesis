\chapter{Literature Survey}
\label{chap:bib}

Since this chapter would normally contain your literature review and
background sections, it seems like a good time to discuss citations
and bibliographies.  The file {\tt thesis.tex} is set to use Harvard
style citations, using the \verb|natbib| package.
The following sections describe how to use this package effectively.

\section{Reference Lists}

The {\tt otagothesis} package automatically renames your selected
bibliography to ``References'', since in Computer Science we only list
works that are actually referred to in the text.  We use the Harvard
style of citations as implemented by the \verb|natbib| package, which
is part of the TeTeX distribution of LaTeX (i.e.\ the one found on
most versions of Linux).

To get all references and citations correct, you need to perform the
\LaTeX\ stanza (see Subsection~\ref{subsec:trouble}): {\tt pdflatex,
bibtex, pdflatex, pdflatex}.  The makefile does this for you when you type
{\tt make}.

Every time you use a \verb|\citet| or \verb|\citep| command, a
bibliographic entry
will be added to your reference list.  On the whole, BibTeX will get
the styles correct, but you should definitely check this before
submitting a final copy; BibTeX bibliographies are notoriously
difficult to spell-check and some strange things can happen with
capitalisation.

\section{Creating a Bibliography File}

If you look at the file {\tt thesis.bib} you will find some
instructions on how to create a BibTeX database, and some sample
entries.  The makefile and thesis template are set up so as to
{\em expect} to see a file called {\tt thesis.bib} (your bibliographic
database file) and will generate an error message if there isn't one.
You can find a large number of Computer Science citations in
BibTeX format on the Collection of Computer Science Bibliographies
Homepage, at {\tt http://liinwww.ira.uka.de/bibliography/}.

A bibliographic entry looks something like this:

\linespread{1} \small
\begin{quote}
\begin{verbatim}
@InProceedings{amir97,
  author = 	 {Amihood Amir and Ronen Feldman and Reuven Kashi},
  title = 	 {{A New and Versatile Method for Association Generation}},
  booktitle = 	 {{Principles of Data Mining and Knowledge Engineering;
                   First European Symposium, PKDD'97}},
  OPTcrossref =  {},
  OPTkey = 	 {},
  editor = 	 {Jan Komorowsk and Jan Zytkow},
  OPTvolume = 	 {},
  OPTnumber = 	 {},
  OPTseries = 	 {},
  year = 	 {1997},
  OPTorganization = {},
  publisher = {Springer-Verlag},
  address = 	 {Trondheim, Norway},
  month = 	 {June},
  pages = 	 {221--231},
  OPTnote = 	 {},
  OPTannote = 	 {}
}
\end{verbatim}
\end{quote}

\linespread{1.3} \normalsize
Separate authors with the word ``and''.  The title is in double
braces to maintain the capitalisation exactly as written, as is the
title of the conference proceedings.  The very first part of the entry
(amir97) must be a unique key---this is what you will use when
referencing the work in your document, so make it something easy to
remember.  Most people use something like ``first\_author\_year'' with
a letter tacked on the end to avoid duplicates (e.g. amir97a, amir97b).

Here are some suggestions for creating your own database:
\begin{itemize}
\item Leave optional fields in place, even when they are empty.  You
never know when that information might turn up, so all you have to do
is add it into the correct field and remove the OPT prefix.
\item Use double braces to maintain your own capitalisation.
You can always remove them later using a search-and-replace, but it is
a real pain to go through the database putting them in.
\item Use the emacs macro to create each new entry (M-x bib-TAB-TAB
for a list of options).  You will be thankful for the consistency as
the list grows.
\item Use ``@InProceedings'' for an article in conference proceedings, not
``@Proceedings''.  Use ``@InCollection'' for an article which forms a
stand-alone chapter in a book where each chapter is written by a
different author, not ``@InBook''.
\end{itemize}


\section{Citation Styles}

The most usual form of citation is an ``aside'' (in parentheses, like
this). The time to use it is when you have made a statement which
might be questionable, and you wish to give a source for it. For
instance:
\begin{quote}
Writing a thesis is easy \citep{rountree98}.
\end{quote}
This type of citation is made using the \verb|\citep| command.  It
should be used sparingly, because it lends itself most to sweeping,
general statements (of which you don't want too many!).

The {\tt natbib} package gives you another option with
the \verb|\citet| command.  This enables you to refer directly
to a piece of work, using it as a noun in your sentence.  For
instance:
\begin{quote}
The article by \citet{amir97} describes a method of preprocessing
a database into a trie, so as to make association generation much quicker.
\end{quote}
This style is more versatile, and should be favoured over the use of
\verb|\citep|.

When there are multiple authors, all subsequent citations of that work
will just list the first author followed by ``et al.''.  This will happen\
automatically; for instance:
\begin{quote}
Here is a second citation of \citet{amir97}.
\end{quote}
If for some reason you want the whole list of authors, you can use the
``starred'' form of the command, i.e. \verb|\citep*| or
\verb|\citet*|. 

Sometimes, it will be inelegant to include the year with the
citation; for example when you are referring to a particular article
very frequently in a single section, (perhaps when a chapter forms a
criticism of another piece of work).  For those purposes, use the command
\verb|\citeauthor|.  This will allow a sentence like ``the
book is of no use to anyone, but \citeauthor{rountree98} published it
anyway.''

The command \verb|\citeyear| is similar, but gives just the year,
allowing you to occasionally break style if you wish---and
\verb|\citeyearpar| will give the year with parentheses.  This will
let you write a sentence like ``\citeauthor{amir97}'s article, published
in \citeyear{amir97}, is the most up-to-date treatment of this topic.''

If you wish to add text to the citation (for instance the letters
e.g., or some page numbers), you may do so either at the beginning or
the end.  For the end (as with page numbers), simply add an optional
argument; \verb|\citep[pp97--100]{rountree98}| will produce
\begin{quote}
The section titled ``Are You Bored Yet?'' is particularly useless
\citep[pp97--100]{rountree98}.
\end{quote}
However if you want to add text at the beginning of the citation, you
should use a combination of the \verb|\citetext| and \verb|\citealp|
commands, like this:
\begin{verbatim}
Several works have been cited in this document
\citetext{e.g., \citealp{rountree98,amir97}}.
\end{verbatim}
producing:
\begin{quote}
Several works have been cited in this document
\citetext{e.g., \citealp{rountree98,amir97}}.
\end{quote}

Citing your current thesis is not allowed.  It is acceptable to refer to
previous work of your own. 
